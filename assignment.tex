%=============================================================================================================
\documentclass{assignment}
\coursetitle{Introduction to Cryptography}
\courselabel{CPSC 418}
\exercisesheet{Home Work \#1}{}
\student{Chris Wozniak - 10109820}
\semester{Fall 2015}

%\usepackage[pdftex]{graphicx}
%\usepackage{subfigure}
\usepackage{multicol}

%=============================================================================================================
\begin{document}

	\begin{center}
		\renewcommand{\arraystretch}{2}
		\begin{tabular}{|c|c|c|} \hline
			Problem & Marks \\ \hline \hline
			1 & 	\\ \hline
			2 & 	\\ \hline
			3 & 	\\ \hline
			4 & 	\\ \hline
			5 & 	\\ \hline
			6 & 	\\ \hline
			7 & 	\\ \hline \hline
			Total & \\ \hline
		\end{tabular}
	\end{center}
	\newpage
%=============================================================================================================
	\begin{problemlist}
		\pbitem Substitution
		\begin{problem}
			\textbf{Compute the plaintext from the provided ciphertext. Code created 
			for solution attached to assignment.}
			\begin{answer}\
				\begin{enumerate}
					\item \underline{Letter Frequency Analysis:} 
					\begin{multicols}{3}
						\begin{itemize} %[font=\consolas]
						%\begin{verb}
							\item[]( X $|$ 136) 
							\item[]( O $|$ 85 ) 
							\item[]( J $|$ 72 ) 	 
							\item[]( R $|$ 59 )
							\item[]( P $|$ 57 ) 	 
							\item[]( T $|$ 57 )
							\item[]( U $|$ 57 ) 	 
							\item[]( I $|$ 56 )
							\item[]( C $|$ 47 ) 	 
							\item[]( A $|$ 46 )
							\item[]( F $|$ 31 ) 	 
							\item[]( N $|$ 31 )
							\item[]( K $|$ 29 ) 	 
							\item[]( E $|$ 27 )
							\item[]( Z $|$ 24 ) 	 	 
							\item[]( D $|$ 20 ) 
							\item[]( L $|$ 19 ) 	 
							\item[]( Q $|$ 17 )
							\item[]( S $|$ 16 ) 	 
							\item[]( G $|$ 14 )
							\item[]( V $|$  9 ) 	 
							\item[]( M $|$  8 )
							\item[]( H $|$  5 ) 	 
							\item[]( B $|$  3 )
							\item[]( Y $|$  3 ) 	 
							\item[]( W $|$  2 )
						%\end{verb}
						\end{itemize}
					\end{multicols}
					
					\item \underline{Decrypted plaintext:}
	
					\par Case was twenty-four. At twenty-two he'd been a cowboy, a rustler, one of 
					the best in the sprawl. He'd been trained by the best, by McCoy Pauley and Bobby Quine, 
					legends in the biz. He'd operated on an almost permanent adrenaline high, a by-product 
					of youth and proficiency. Jacked into a custom cyberspace deck hat, projected his 
					disembodied consciousness into the consensual hallucination that was the matrix. A thief, 
					he'd worked for other wealthier thieves. Employers who provided the exotic software 
					required to penetrate the bright walls of corporate systems, opening windows into rich 
					fields of data.\\
	
					\par He'd made the classic mistake, the one he'd sworn he'd never make: he stole from 
					his employers. He kept something for himself and tried to move it through a fence in 
					amsterdam. He still wasn't sure how he'd been discovered, not that it mattered. Now he'd 
					expected to die then but they only smiled, of course he was welcome. They told him welcome 
					to the money and he was going to need it, because still smiling they were going to make 
					sure he never worked again.\\
	
					\par They damaged his nervous system with a wartime "Russian Mycotoxin" \\
					
					\item \underline{Who originally wrote the plaintext?} (Bonus)
					\begin{center}
						\par William Gibson, Neuromancer
					\end{center}
				\end{enumerate}
			\end{answer}
		\end{problem}
		\newpage
%=============================================================================================================	
		\pbitem Compound Cipher Theory
		\begin{problem}
			The shift cipher.
			\begin{answer}
				\begin{enumerate}
					\item  Give a formal proof that multiple shifts results in a shift cipher.
						\begin{enumerate}
							\item \underline{eK1 + ek2 = ek1+k2}
							\item \underline{multiple enumeration}
						\end{enumerate}
					\item Multiple encryption
						\begin{enumerate}
							\item \underline{Vigenere key pairs}
							\item \underline{Vigenere double encryption}
						\end{enumerate}
				\end{enumerate}
			\end{answer}
		\end{problem}
		\newpage
%=============================================================================================================	
		\pbitem Entropy
		\begin{problem}
			\begin{answer}
				\begin{enumerate}
					\item \underline{Entropy}
					\item \underline{Maximal}
					\item \underline{Maximal Value of H(x)}
				\end{enumerate}
			\end{answer}
		\end{problem}
		\newpage
%=============================================================================================================	
		\pbitem Password lengths
		\begin{problem}
			\begin{answer}
				\begin{enumerate}
					\item \underline{Total number of ASCII encodings}
					\item \underline{Unusable ascii codes}
						\begin{enumerate}
							\item \underline{key space sizes}
							\item \underline{percentage of ascii codes are permissible}
						\end{enumerate}
					\item \underline{Entropy of keyspace}
					\item \underline{Entropy of restricted keyspace}
					\item \underline{Entropy 128}
						\begin{enumerate}
							\item \underline{all ascii chars}
							\item \underline{ascii lower case restricted}
						\end{enumerate}
				\end{enumerate}
			\end{answer}
		\end{problem}
		\newpage
%=============================================================================================================	

%=============================================================================================================	

%=============================================================================================================	
	\end{problemlist}
\end{document}
